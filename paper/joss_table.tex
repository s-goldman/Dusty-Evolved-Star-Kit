%% using aastex version 6.3
\documentclass[onecolumn]{aastex63}
% \usepackage[papersize={10in,7in}, left=0.1in,right=0.1in,top=1in,bottom=0.1in]{geometry}
\usepackage[papersize={7.85in,5.0in}, top=2.5cm, bottom=-0.5cm, left=1.0cm, right=0.25cm]{geometry}

\usepackage{graphicx}
\usepackage{ascii}
\usepackage{amsmath}
\usepackage{multirow}

\begin{document}


\vspace{3cm}
\begin{table*}
    \centering
    \caption{The model grids available with this version. Shown is the name, size, chemical type of either oxygen (O) or carbon (C), the atmospheric model, and a brief description.}
    % \setlength{\tabcolsep}{0.76em}
    \begin{tabular}{ l r c c c c c}
    \hline
    Grid name & Size & Type & Atmospheric model & Optical constants & References \\
    \hline \hline
    Oss-Orich-aringer & 2,000 & O & COMARCS & Warm silicates & 1, 6\\
    Oss-Orich-bb & 2,000 & O & Black body (BB) & Warm silicates & 6\\
    Crystalline-20-bb & 2,000 & O & BB & 80\% warm silicates, 20\% crystalline silicates & 4, 6\\
    corundum-20-bb & 2,000 & O & BB & 80\% warm silicates, 20\% corundum silicates &  2, 6\\
    big-grain & 2,000 & O & BB & Warm silicates with higher maximum dust grain size of 0.35 &  6\\
    fifth-iron & 500 & O & BB & 80\% warm silicates, 20\% iron grains &  3, 6\\
    half-iron & 500 & O & BB & 50\% warm silicates, 50\% iron grains &  3, 6 \\
    one-fifth-carbon & 500 & O & BB & 80\% warm silicates, 20\% carbonaceous grains &  6, 7\\
    arnold-palmer & 500 & O & BB & 50\% warm silicates, 50\% carbonaceous grains &  6, 7\\
    Zubko-Crich-aringer & 2,000 & C & COMARCS & Amorphous carbon grains &  1, 7\\
    Zubko-Crich-bb & 2,000 & C & BB & Amorphous carbon grains &  7\\
    H11-LMC & 90,899 & C & COMARCS & Dust-growth grid with 1/2 solar metallicity &  5\\
    H11-SMC & 91,058 & C & COMARCS & Dust-growth grid with 1/5 solar metallicity &  5\\
    J1000-LMC & 85,392 & C & COMARCS & Dust-growth grid with 1/2 solar metallicity &  5\\
    J1000-SMC & 85,546 & C & COMARCS & Dust-growth grid with 1/5 solar metallicity & 5\\
    \hline
    \vspace{-0.3cm}
    \end{tabular}
    \begin{flushleft}
    {{References}: $^1$Aringer et al. (2016), $^2$Begemann et al. (1997), $^3$Henning et al. (1995), $^4$Jaeger et al. (1998),  $^5$Nanni et al. (2019), $^6$Ossenkopf et al. (1992), $^7$Zubko et al. (1996)}
    \end{flushleft}

\end{table*}

\thispagestyle{empty}


\end{document}
